% !TeX program = lualatex
\documentclass[a5paper, 12pt]{article}
%\usepackage[a5,center,off, layoutvoffset=3mm,layouthoffset=3mm]{crop}
\usepackage[layout=a5paper,layoutvoffset=0mm,layouthoffset=0mm,showframe=false,showcrop=false]{geometry}

\usepackage{blindtext}
\usepackage{tikz}
\usetikzlibrary{calc}
\usepackage[utf8]{inputenc}
\usepackage{multicol}
\usepackage{pdflscape}
\usepackage{longtable}
\usepackage{xcolor}
\setlength{\columnsep}{2cm}
\usepackage{fancyhdr}
\usepackage{everypage}
\usepackage{lipsum}
\usepackage{array}
\usepackage{lipsum}
\usepackage{tikz}
\usetikzlibrary{fit,shapes.geometric}
\usepackage{setspace}
\usepackage{adjustbox}
\usepackage{fontspec}
\usepackage{titlesec}
\usepackage{titling}
\usepackage{rotating}
\usepackage{pdfpages}
\newfontfamily\headingfont[]{parisienne}
\usepackage{titlesec}
\titleformat{\section}[block]{\Large\bfseries\filcenter}{}{1em}{}
\newcommand\YUGE{\fontsize{48}{60}\selectfont}
\usepackage{afterpage}

\newcommand\blankpage{%
	\null
	\thispagestyle{empty}%
	\addtocounter{page}{-1}%
	\newpage}

\begin{document}
\pagenumbering{gobble}	
 \cleardoublepage

%%%%%%%%%%%%%%%%%%%%%%%%%%%%%%%%%%%%%%
		
%%%%%%%%%%%%%%%%%%%%%%%%%%%%%%%%%%%%%%

	\begin{titlepage}
		\pagestyle{empty}
		\newgeometry{top=1cm, left=2cm,bottom=0.1cm}
		\noindent
			\begin{tikzpicture}[remember picture,overlay, every node/.style={anchor=south west,inner sep=0pt}, x=1mm,y=1mm]   
			\node[opacity=1.0] (fig1) at (-3.29cm,-15cm)
				{\includegraphics[trim={12.5cm 17.2cm 0 0.0cm},clip, scale=1.35]{ornament.eps}};
			\end{tikzpicture}
			\par
			\noindent
			\fontspec{parisienne}
			\vspace*{2.0cm}
			\begin{center}
				\YUGE Emma och Johnnys bröllop
			\end{center}
			%\vfill
			\vspace*{7.5cm}
			\noindent
			{\huge \textsf{Vigselprogam}}
			\vskip\baselineskip
			\noindent
			\Large \textsf{6 Oktober 2018}
		\end{titlepage}

	\restoregeometry
	\newgeometry{top=1.5cm,bottom=1cm}
	
	
\begin{center}
		
		
	\section{Klockringning}
	\vspace{0.4cm}
	
	\section{Inledningsmusik}
	\vspace{0.4cm}
	
	\section{Psalm - På bröllopsdagen ber vi}
	\vspace{-0.3cm}
	\textit{Melodi: Den blomstertid nu kommer} \\
	\vspace{0.2cm}
	På bröllopsdagen ber vi    \\
	till dig som kärlek är.    \\
	Du lever mitt ibland oss   \\
	i glädjen nu och här.      \\
	Vi prisar dig som skapar   \\
	och känner allt som sker.  \\
	Förnyare av livet,         \\
	till dig, vår Gud, vi ber. \\
	\vspace{0.5cm}
	Om kärlek som får djupna, \\
	förlåta och förstå,          \\
	om kärlek genom dagar        \\
	då mörkret skall förgå,      \\
	om goda, ljusa dagar         \\
	då lyckan vänligt ler,       \\
	om trohet i vår kärlek,      \\
	till dig, vår Gud, vi ber.   \\
	\vspace{0.2cm}

	
	\section{Inledningsord och bibelläsning}	
	\vspace{0.4cm}
	
	\section{Solosång}
	\vspace{-0.4cm}
	\textit{För dig, Lars Winnerbäck}
	
	
	\section{Vigselakt}
	\vspace{0.4cm}
	
	\section{Psalm 84 - Vi lyfter våra hjärtan}
	Vi lyfter våra hjärtan, till dig, o Fader vår, i tacksamhet och lovsång och bön för dessa två.      \\
	\vspace{0.2cm}
	Vi ber dig, låt dem lämna sig själva i din hand och hjälp dem hålla löftet som de har gett varann.   \\
	\vspace{0.2cm}
	Vi tackar för den glädje som du har gett åt dem. Välsigna deras framtid, välsigna deras hem.         \\
	\vspace{0.2cm}
	Tack, Fader, för den kärlek som gjorde dem till ett, det största du som gåva och ansvar åt oss gett. 

	\vspace{0.2cm}
	\section{Förbön}	
	\vspace{0.4cm}
	
	\section{Fader vår}
	\vspace{0.4cm}
	
	\section{Välsignelsen}	
	\vspace{0.4cm}

	\newpage
		\section{Psalm 791 - Du vet väl om att du är värdefull}
		Du vet väl om att du är värdefull   \\
		Att du är viktig här och nu         \\
		Att du är älskad för din egen skull \\
		För ingen annan är som du           \\
		\vspace{0.2cm}
		Det finns alltför många som vill tala om \\
		Att du bör vara si och så                \\ 
		Gud Fader själv, han accepterar dej ändå \\
		Och det kan du lita på                   \\
		\vspace{0.2cm} 
		Du vet väl om …                          \\
		\vspace{0.2cm}
		Du passar in i själva skapelsen            \\  
		Det finns en uppgift just för dej          \\ 
		Men du är fri att göra vad du vill med den \\ 
		Säga ja eller nej                          \\
		\vspace{0.2cm}   
		Du vet väl om …                            \\
	\vspace{0.4cm}

	\section{Tal av prästen}
	\vspace{\fill} 
	
	\section{Utgångsmusik}
\vspace{\fill} 

	\section{\hfil Medverkande \hfil}
	Vigselförrättare: Johan Lamberth \\
	Kantor: Vivi Boozon \\
	Solist: Laura Betnér, Fredrik Valdeson \\
	
	\end{center}


\end{document}